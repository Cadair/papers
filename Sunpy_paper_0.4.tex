\documentclass[12pt]{iopart}


%Uncomment next line if AMS fonts required
%\usepackage{iopams}

\begin{document}

\title{SunPy - Python for Solar Physics}

\author{S J Mumford}
\address{Solar Physics \& Space Plasma Research Centre (SP$^{2}$RC), School of 
Mathematics and Statistics, The University of Sheffield, Hicks Building, 
Hounsfield Road, Sheffield, S3 7RH U.K.}

\author{J Bloggs}
\address{Wibble wiblle wiblle}

\ead{sunpy@googlegroups.com}

\begin{abstract}
This paper presents version 0.4 of SunPy a community developed Python package 
for solar physics.

\end{abstract}

\maketitle

\section{Introduction}
SunPy is a project dedicated to providing the tools to process and analyse 
solar data in Python.
The Python programming language has gained much traction as a scientific 
programming language in the last decade.
Projects such as \texttt{NumPy}, \texttt{SciPy}, \texttt{matplotlib} provide 
the core tools 
needed 
for 
scientific analsysis and visualisation, while other packages such as pandas, 
Astropy or scikit-image provide more domain specific functionality.
SunPy aims to provide such domain specific functionality to the solar physics 
community.

The goal of SunPy is to provide a clean, simple to use and well structured 
package that provides the \textit{core} tools for solar physics.
The primary focus of SunPy's development is current providing specialised, 
linked, datatypes that allow the import, processing and visualisation of solar 
data. Currently the datatypes implemented are \texttt{Map}, \texttt{LightCurve} 
and \texttt{Spectrum}, these cover spatial, timeseries and spectral data 
respectively.

In this paper we will provide an overview of the current functionality of the 
\texttt{sunpy} library as well as giving details on the development model and 
community aspects of the SunPy project.

\section{Core Data Types}
One of the primary focus' of the SunPy library is to provide data structures 
that are 
specifically designed for the various types of data generally processed for 
solar physics. Currently \texttt{SunPy} provides three datatypes, each separate 
from one another to process 2D spatial data (Map), 1D temporal series 
(LightCurve) and 1 and 2D spectra (Spectrum and Spectra).

This section will give a brief overview of the \textit{current} functionality 
of each of these modules and then describe the development plan for future 
improvements to these submodules.

	\subsection{Map}
	Map is a 2D spatial data type, primarily used for images of the Sun and 
	inner heliosphere. It provides a wrapper around a data array (numpy 
	ndarray) and gives easy access to standard meta data in the header of the 
	image.
	The \texttt{Map} datatype provides convenience methods for many functions 
	such as, rotation and re-sampling as well as convenience visualisation 
	functions.
	The design of the map submodule is such that each instrument or 
	detector can subclass the parent \texttt{GenericMap} class to implement 
	special meta data handling or other data specific functions. Each subclass 
	of \texttt{GenericMap} can register with the \texttt{Map} factory class and 
	by implementing a method that returns \texttt{True} if the meta data 
	matches meta data for that instrument or detector, the \texttt{Map} factory 
	will automatically return an instance of the specific \texttt{GenericMap} 
	subclass. As of version 0.4 SunPy has \texttt{Map} specialisations for the 
	following instruments: \textit{IRIS} SJI frames, \textit{SDO/AIA} and 
	\textit{HMI}, \textit{SOHO/EIT} and 	\textit{LASCO}, 
	\textit{STEREO/EUVI} 	and \textit{COR}, \textit{YOHKOH/SXT}, 
	\textit{RHESSI}, \textit{PROBA2/SWAP} and\textit{HINODE/XRT}.
	As well as providing the base classes the map submodule provides two 
	collection classes, \texttt{CompositeMap} and \texttt{MapCube}, for 
	temporally and spatially aligned data respectively. \texttt{MapCube} 
	provides methods for animation of its series of \texttt{Map} objects. 
	\texttt{CompositeMap} provides methods for overlaying spatially aligned 
	data, with support for visualisation of images and contour lines overlaid 
	upon each other.
	
	\subsection{Lightcurve}
	
	\subsection{Spectra}

\section{Solar Data Search and Retrieval}

	\subsection{VSO}
	
	\subsection{HEK}
	
	\subsection{HELIO}
	
	\subsection{The SunPy Database}

\section{Visualisation}

\section{Coordinates}

\section{Development and Community}

\section{Future of SunPy}

%%% References
\bibliography{sunpy_paper_0.4}{}

\end{document}

