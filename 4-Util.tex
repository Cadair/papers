\section{Additional functionality}\label{sec:util}
SunPy is meant to provide a consistent environment for solar data analysis. In 
order to achieve this goal SunPy provides a number of additional functions and packages which 
are used by the other SunPy modules and are made available to the user. This section 
briefly describes some of these functions.

%Two modules are described in this Section, namely, solar coordinates (WCS), and 
%the Sun module. 
	
\subsection{World Coordinate System (WCS) Coordinates}\label{ssec:util:wcs}
Coordinate transformations are frequently a necessary task within the solar 
data analysis workflow. Likely the most often used transformation is from 
observer coordinates (e.g., sky coordinates) to a coordinate system which is 
mapped onto the solar surface (e.g., latitude and longitude). This 
transformation is necessary to compare the true physical distance between 
different solar features. This type of transformation is not unique
to solar observations, but is not often considered by astronomical packages
such as the Astropy 
coordinates package. The \texttt{wcs} package in SunPy implements the World Coordinate 
System (WCS) for solar coordinates as described by \citep{thompson2006}. The 
transformations currently implemented are those most useful 
for most solar data analysis, namely converting from Helioprojective-Cartesian 
(HPC) to Heliographic (HG) coordinates. HPC describes the positions on 
the Sun as angles measured from the center of the solar disk (usually in 
arcseconds) using Cartesian coordinates (X, Y). This is the coordinate system 
most often defined in solar imaging data (see for example, images from 
\textit{SDO}/AIA, \textit{SOHO}/EIT, and \textit{TRACE}). 
HG coordinates express positions on the Sun using longitude and latitude on 
the solar sphere. There are two standards for this coordinate system:
Stonyhurst-Heliographic, where the origin is at the intersection of the solar 
equator and the central meridian as seen from Earth, and 
Carrington-Heliographic, which is fixed to the Sun and does not depend on Earth. The 
implementation of these transformations pass through a common coordinate system 
called Heliocentric-Cartesian (HCC), where positions are expressed in true 
(de-projected) physical distances instead of angles on the celestial sphere.
These transformations require some knowledge of the location of the observer 
which is usually provided by the image header. In the case where it is 
not, the observer is assumed to be at Earth. Listing \ref{code:wcs_code} shows 
some examples of coordinate transforms carried out in SunPy using the 
\texttt{wcs} utilities. 

\begin{listing}[H]
\begin{minted}[bgcolor=bg]{pycon}
>>> from sunpy import wcs
>>> wcs.convert_hg_hpc(10,53)
(100.49244115330731, 767.97438321917502)
# Convert that position back to heliographic coordinates
>>> wcs.convert_hpc_hg(100.49, 767.97)
(9.9996521808465175, 52.999563684874893)
# Try to convert a position which is not on the Sun to HG
>>> wcs.convert_hpc_hg(-1500,0)
sunpy/wcs/wcs.py:180: RuntimeWarning: invalid value encountered in sqrt
  distance = q - np.sqrt(distance)
(nan, nan)
Convert sky coordinate to a position in HCC
>>> wcs.convert_hpc_hcc(-300,400, z = True)
(-216716967.63331246, 288956420.9477042, 594364636.2208252)
\end{minted}
\caption{Using the \texttt{wcs} subpackage.}
\label{code:wcs_code}
\end{listing}

\subsubsection{Sun}\label{ssec:util:sun}
The purpose of the \texttt{sun} subpackage is to provide solar-specific data such as ephemerides and
solar constants. The main namespace contains a number of functions which provide solar
ephemerides such as the Sun to Earth distance, solar-cycle number, the mean 
anomaly, etc.
All of these functions take a time as their input which can be provided in a format
compatible with \texttt{sunpy.time.parse\_time()}. 
A future version of SunPy will integrate with the Astropy coordinates package and will 
likely make use of an ephemeris package.

%ays: I removed this as a separate subsection because constants is part of sun...
%\subsubsection{constants}\label{ssec:util:const}
The \texttt{sun.constants} module provides a number of solar-related 
constants in order to provide consistency in the calculations of derived solar 
values both within the SunPy code base but also to the user. Every solar 
constant is provided as a \texttt{Constant} object as defined by Astropy. Each 
\texttt{Constant} object defines a \texttt{Quantity}, a number associated with a unit, along with 
the constant's provenance (i.e., reference) as well as its uncertainty. Using 
Astropy's {Quantity} objects, any solar constant can easily be converted between 
different units as can be seen in Listing~\ref{code:constants_code} as well
as between the SI or cgs unit systems with a single command.  As these objects inherit from 
NumPy's \texttt{ndarray}, they work well with standard representations of numbers.
For convenience a number of shortcuts to frequently used constants are provided 
directly when importing the module. A larger list of constants can be 
accessed through an interface modelled on that provided by the SciPy constants 
package and is available as a dictionary called \texttt{physical\_constants}. 
To view them all quickly, a \texttt{print\_all()} function is available.

\begin{listing}[H]
\begin{minted}[bgcolor=bg]{pycon}
>>> from sunpy.sun import constants
>>> print(sun.mass)
  Name   = Solar mass
  Value  = 1.9891e+30
  Error  = 5e+25
  Units  = kg
  Reference = Allen's Astrophysical Quantities 4th Ed.
# Verify the average density of the Sun and convert to cgs
>>> (constants.mass/constants.volume).cgs
<Quantity 1.40851154227 g / (cm3)>
# Search for the age of the Sun
>>> constants.find('age')
['age', 'average angular size', 'average density', 'average intensity']
>>> constants.value('age'), constants.unit('age')
(4600000000.0, Unit("yr"))
\end{minted}
\caption{Using the \texttt{sun.constants} module.}
\label{code:constants_code}
\end{listing}
	
\subsubsection{Instruments}\label{ssec:util:inst}
In addition to providing support for instrument-specific solar data via the main data 
classes \texttt{Map}, \texttt{LightCurve} and \texttt{Spectrum}, 
some instrument-specific functions may be found within the \texttt{instr} subpackage. 
These functions are generally those that are unique to one particular solar instrument, 
rather than of general use, such as a function to construct a \textit{GOES} flare event list 
or a function to query the \textit{LYRA} timeline annotation file. Currently, some support is included
 for the \textit{GOES}, \textit{LYRA}, \textit{RHESSI} and \textit{IRIS} instruments, while future developments 
 will include support for additional missions. Ultimately, it is anticipated that solar
  missions requiring a large suite of software tools will be supported via a separately 
  maintained package that is affiliated with SunPy.

