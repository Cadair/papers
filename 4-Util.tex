\section{Utilities}
SunPy is meant to provide a consistent environment for solar data analysis. In order to
achieve this goal SunPy provides a number of utility functions which are used by the other
SunPy modules and are made available to the user. Two modules are described in this Section, 
namely, solar coordinates (coords), and the Sun module (sun). 
	
\subsection{Coordinates}
Coordinate transformations are frequently a necessary task within the solar data analysis
workflow. Likely the most often used transformation is from observer coordinates (e.g. 
sky coordinates) to a coordinate system which is mapped onto the solar surface (e.g.
latitude and longitude). This transformation is necessary to compare the true physical
distance between different solar features. This type of transformation is not unique
to solar observations (e.g. planetary) but it is not provided by Astropy coordinates 
package. 

\subsubsection{Sun}
The purpose of this module is to provide solar-specific data

\subsubsection{constants}
The purpose of the constants package is to provide a number of solar-related constants 
in order to provide consistency in the calculations of derived solar values both within the
SunPy code base but also to the user. Every solar constants is provided as a Constant objects as defined by Astropy. 
Each Constant object defines a Quantity, a number associated with a unit, along with the constant's provenance
(i.e. reference) as well as its uncertainty. Using Astropy's Quantities, any solar constant
can easily be converted between different units as can be seen in Figure~\ref{fig:constants_code} as well
as between SI or cgs with a single command.  As these objects inherit from Numpy's ndarray,
they work well with standard representations of numbers.
For convenience a number shortcuts to frequently used constants are provided directly when importing
the module. A larger list of constants can be accessed through an interface modelled 
on that provided by the SciPy constants package. To view all of the available constants
a print_all() function can be used.   
	