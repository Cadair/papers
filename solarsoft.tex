\section{Comparison to SolarSoft}
This section compares the SunPy library that is described in Sections 
\ref{sec:DataTypes}--\ref{sec:util} with the current de-facto standard solar 
physics data analysis library: SolarSoft (SSW) \citep{freeland1998}.
SolarSoft is a set of integrated software libraries, databases, and system 
utilities which provide a common programming and data-analysis environment for 
solar physics. SSW is modular in nature, which allows users to add required 
packages which contain specialised software for different instrument or tools.

SSW relies upon IDL (Interactive Data 
Language), a commercial and closed-source interactive data analysis environment 
sold by Exelis, a global aerospace, defence, information and services company. 
The SSW `gen' package provides the base analysis capabilities such as time-series
analysis, time conversions, spectral fitting, image display, and file 
I/O. It is composed of 89\% IDL code while the remainder is mostly Perl (5.4\%) 
and csh scripts (3.8\%). 
According to data generated using David A. Wheeler's `SLOCCount'\footnote{\url{http://www.dwheeler.com/sloccount/}}, SSW/gen is composed of 
289,724 lines of code with a total approximate value of \$10.4M (based on the 
Basic COCOMO model \citep{_cocomo_2014}). 
%stuart - This data was calculated using...} If we can add them to the repository 
%will improve the quality of the paper and we can always repeat the calculation in the future.
%schriste - i don't know what you mean by this 
Individual missions and instruments provide additional optional packages to 
support and process their data. For example, the SDO package \citep{sdo} is 
100\% IDL code and is 11k lines of code (\$360k) and the SOHO package 
\citep{soho} is 100k lines of code (\$3M) and is 99.9\% IDL code. SSW is hosted 
and distributed by the Lockheed Martin Solar and Astrophysics Laboratory.
%The RHESSI package is 99.97\% IDL code and is 145k total lines of code (\$5M)
While SSW itself is open-source and freely available, the IDL core is not. The cost 
of IDL is prohibitive for many undergraduates as well as institutions 
with limited means.
Even though SSW's source is open, the development of the source is primarily closed,
with no open issue tracker or established mechanism for community contributions.

One of SunPy's key aims is to provide a free and modern alternative to the 
SolarSoft (gen) library. This is made possible by the recent rise of the 
scientific Python community. Python, as described in Section \ref{sec:Intro}, 
is a general purpose high-level programming language which is free to use and 
cross-platform and powerful programming language.
It therefore has a very large user community which extends far outside of the 
solar physics or astronomy communities. It is very extensible with C, C++, 
FORTRAN or even IDL. Many books and on-line documentation is available. For 
example, at the time writing a search 
%
%stuart - \footnote{We need to be careful with this! IDL means more things that our 
%beloved Interactive Data Language. If we search for that and programming 
%then we get just 7 books! - from which 5 are just the same. I don't know how we 
%could make a more accurate search.
%If we script it it would be amazing!  Also, I would just provide the order
%of magnitude, and not a exact number, i.e. 190 and 1800 results}
% 
% schriste - I think we defined IDL properly. I was being generous with the book search for IDL
on Amazon for books on IDL programming generated $189$ results while a search 
for Python yielded $1\,788$ results. Python has a strong emphasis on 
readability a concept which is well-aligned with the scientific endeavour.
Additionally, Python is increasingly being taught in universities, which means 
that research projects can begin quickly without the need for students to learn 
a new (and relatively niche) language. Finally, knowledge of Python is a skill 
which is useful and well compensated outside of the scientific community 
(average salary of \$110k in the Washington, DC, area according to indeed.com at 
the time of writing). Python continues to see increased use in the astronomy 
community \citep{greenfield2011} which has similar goals and requirements as 
the solar physics community. Finally, Python integrates well with many 
technologies such as web servers \citep{dolgert2008}, SQL databases, 
interactive and shareable notebook-based computing \citep{perez2007}. For all 
of these reasons, Python is an ideal platform for a modern data-
analysis environment for solar physics.

SunPy is built on top of the core scientific Python packages namely 
\texttt{NumPy}, \texttt{SciPy}, \texttt{matplotlib}. These core packages 
provide capabilities surpassing that provided by IDL for array manipulation and 
2D plotting \citep{greenfield2011}. Providing similar capabilities as SSW/gen 
in Python is made easier by leveraging the diverse Python community. For 
example, SunPy makes use of the \texttt{pandas} package to provide 
high-performance, easy-to-use data structures and data analysis tools for time 
series data\citep{mckinney2012}. \texttt{pandas} was originally developed for 
quantitative analysis of financial data and has since grown into a generalised 
data analysis package. SunPy also relies on and is coordinating with the 
\texttt{Astropy} package \citep{theastropycollaboration2013}. \texttt{Astropy} is still in 
active development but already provides many relevant capabilities such as 
full-featured support for FITS files, absolute date and time conversions, 
units and unit conversions, and model-fitting. Pandas (134k lines of code) 
and Astropy (90k lines of code) allow SunPy to leverage a total value of \$7.6M 
worth of development which is close to the \$10.4M for SSW/gen assuming a rough 
equivalency between a line of Python code and a line of IDL code. 