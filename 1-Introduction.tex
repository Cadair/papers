\section{Introduction}\label{sec:Intro}

Science is driven by the analysis of data. Modern advances in sensor
technology, combined with the availability of inexpensive storage has led to
rapid increases in amount of data available to scientists in almost
every discipline.  Solar physics is no exception to this trend. For example, NASA's
Solar Dynamics Observatory (SDO) satellite, launched in February 2010,
produces over 1 TB of catalogued and stored data per day [REF]. Managing and
analysing this mountain of data
requires increasingly sophisticated software tools.  Key qualities
that these tools should have include robustness, ease of use, a
transparent development history, ease of modification, and conformance
to standards.  Software with these qualities provides a strong
foundation can be responsive to the needs of the community as data
volumes grow and science questions evolve.
%schriste - the opening paragraph still needs work
%ji - I had a go.

SunPy aims to provide a free, open-source and openly developed software package 
for the analysis and visualisation of solar data. SunPy makes use of the Python 
programming language and the breadth of high-quality science-focused packages 
written in that language. Python is a general-purpose, 
powerful and easy-to-learn high-level programming language.
% RJH: Can we cut the next sentence?  It seems superfluous and top 10 is rather
% meaningless.
According to the 2014 TIOBE Index\footnote{\url{http://www.tiobe.com/index.php/content/paperinfo/tpci/index.html}},
 it is one of the top ten most popular programming languages in the world 
and is widely used outside of scientific fields in areas like web development, education 
and `big data' analytics.


The development of a package such as SunPy in Python is made possible by the 
rich ecosystem of scientific packages available in Python. Core packages such as \texttt{NumPy}, \texttt{SciPy} and \texttt{matplotlib} 
provide the basic functionality expected of a scientific programming language,
e.g., array manipulation, core numerical algorithms and visualisation. 
Building upon these foundations, packages such as \texttt{astropy}, \texttt{pandas} and 
\texttt{scikit-image} provide more domain-specific functionality.

SunPy is designed to be a clean, simple-to-use, and well 
structured package that provides the \textit{core} tools for data analysis in 
solar physics. The primary focus of SunPy's early development is to provide 
specialised data types that allow the acquisition, processing and 
visualisation of all types of solar data.

The purpose of this paper is to provide an overview of SunPy's current 
capabilities, an overview of the project's development model, community aspects of the 
SunPy project, as well as future plans. The latest release of SunPy, version 0.4,
can be downloaded from \href{http://sunpy.org}{www.sunpy.org} or can be
installed using the \href{http://pypi.python.org/pypi}{Python package index}.
