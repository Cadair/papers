\section{Introduction}
SunPy is a project dedicated to providing the tools to process and analyse 
solar data in Python.
The Python programming language has gained much traction as a scientific 
programming language in the last decade.
Projects such as \texttt{NumPy}, \texttt{SciPy}, \texttt{matplotlib} provide 
the core tools needed for scientific analysis and visualisation, while other 
packages such as pandas, Astropy or scikit-image provide more domain specific 
functionality. SunPy aims to provide such domain specific functionality to the 
solar physics community.

The goal of SunPy is to provide a clean, simple to use and well structured 
package that provides the \textit{core} tools for solar physics.
The primary focus of SunPy's development is current providing specialised, 
linked, datatypes that allow the import, processing and visualisation of solar 
data. Currently the datatypes implemented are \texttt{Map}, \texttt{LightCurve} 
and \texttt{Spectrum}, these cover spatial, timeseries and spectral data 
respectively.

In this paper we will provide an overview of the current functionality of the 
\texttt{sunpy} library as well as giving details on the development model and 
community aspects of the SunPy project.