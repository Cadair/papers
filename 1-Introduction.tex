\section{Introduction}
Modern solar physics, similar to astrophysics, requires increasingly complex software 
tools both for the retrieval as well as the analysis of data. The Sun is the most well-observed 
star. As such, solar physics is unique in its ability to access large amounts of high 
resolution ground- and space-based observations of the Sun at many different wavelengths 
and spatial scales with high time cadence. Modern solar physics, similar to astrophysics, 
therefore requires increasingly complex software tools, both for the retrieval and the 
analysis of data. For example, NASA's Solar Dynamics Observatory (SDO) satellite records 
over 1 TB of data per day all of which is telemetered to the ground and available for 
analysis. As a result, scientists have to process large volumes of complex data products. 
In order to make meaningful advances in solar physics, it is important for the software 
tools to be standardized, easy to use, and transparent, so that the community can build 
upon a common foundation.

Currently, most solar data analysis is performed with a large library of routines called 
SolarSoft [SSW]. SolarSoft is a set of integrated software libraries, databases, and 
system utilities which provide a common programming and data analysis environment for
solar physics. It is overwhelmingly an environment which relies upon IDL (Interactive Data Language),
a commercial and closed-source interactive data analysis environment sold by 
Exelis a global aerospace, defense, information and services company. The SSW gen package
provides the base analysis capabilities such as time series analysis, time conversions,
spectral fitting, image display, and file I/O. It is composed of 89\% IDL code while the remainder
is mostly perl (5.4\%) and csh scripts (3.8\%). According to data generated using 
David A. Wheeler's 'SLOCCount', SSW/gen is composed of 289,724 lines of code with a total approximate
value of \$10.4M (based on the Basic COCOMO model). 
Individual missions and instruments provide additional optional packages. For example, 
the SDO package is 100\% IDL code and is 11k lines of code (\$360k) and the SOHO package is
100k lines of code (\$3M) and is 99.9\% IDL code. SSW is hosted and distributed by the 
Lockheed Martin Solar and Astrophysics Laboratory.

%The RHESSI package is 99.97\% IDL code and is 145k total lines of code (\$5M)

While SSW is open-source and freely available, the IDL core is not. In addition, the development
of SolarSoft is neither open nor transparent to the public. Also IDL is fairly limited in 
its ability to interact with other languages. One of SunPy's key aims is to 
provide a free and modern alternative to the SolarSoft (gen) library. This is made possible
by the recent rise of the scientific Python community. Python is general purpose 
high-level programming language. It is free to use, cross-platform, and has a strong
emphasis on readability a concept which is well-aligned with the scientific endeavor.
Python has a very large user community which extends far outside of the solar physics
or astronomy communities. It is very extensible with C, C++, Fortran or even IDL. It is 
easy to learn. Many books and on-line documentation is available. For example at the time
of this writing a search on Amazon for books on IDL programming generated 189 results while a search
for Python yielded 1,788 results. Additionally, students may already have learned Python during their 
undergraduate which means that research projects can begin quickly without the need for 
the student to learn a new (and relatively niche) language. Finally, knowledge of Python is a skill which is useful
and well compensated outside of the scientific community (average salary of \$110k in Washington DC area according
to indeed.com at the time of writing). Finally, the astronomy community has begun to
move away from IDL to Python. For all of these reasons, Python is an ideal platform for
the generation of solar physics data analysis environment.

SunPy is built on top of the core scientific Python packages namely \texttt{NumPy}, \texttt{SciPy}, 
\texttt{matplotlib}, and \texttt{Pandas}. As of the release 0.4, SunPy is also leveraging the 
newly release \texttt{Astropy} package (0.3). Leveraging all of these packages has allowed 
the development of SunPy to provide many capabilities without very much effort. The design
philosophy of SunPy is to provide a clean, simple to use and well structured 
package that provides the \textit{core} tools for solar physics.
The primary focus of SunPy's development early development is to providing specialised, 
linked, datatypes that allow the import, processing and visualisation of solar 
data. Currently the datatypes implemented are \texttt{Map}, \texttt{LightCurve} 
and \texttt{Spectrum}, these cover spatial, timeseries and spectral data 
respectively.

The purpose of this paper is to provide an overview of SunPy's current capabilities, an 
overview of the development model and community aspects of the SunPy project
as well as future plans. The latest release of SunPy is available in 
PyPI and can be installed in the usual manner.
