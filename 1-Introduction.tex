\section{Introduction}
Modern solar physics, similar to astrophysics, requires increasingly 
complex software tools both for the retrieval as well as the analysis 
of data. The Sun is the most well-observed star. As such, solar 
physics is unique in its ability to access large amounts of high 
resolution ground- and space-based observations of the Sun at many 
different wavelengths and spatial scales with high time cadence. 
Modern solar physics, similar to astrophysics, therefore requires 
increasingly complex software tools, both for the retrieval and the 
analysis of data. For example, NASA's Solar Dynamics Observatory 
(SDO) satellite records over 1 TB of data per day all of which is 
telemetered to the ground and available for analysis. As a result, 
scientists have to process large volumes of complex data products. In 
order to make meaningful advances in solar physics, it is important 
for the software tools to be standardized, easy to use, and 
transparent, so that the community can build upon a common foundation.

Currently, most solar data analysis is performed with a large library 
of routines called SolarSoft, also referred to as SSW 
\cite{freeland1998}. SolarSoft is a set of integrated software 
libraries, databases, and system utilities which provide a common 
programming and data analysis environment for solar physics. It is 
overwhelmingly an environment which relies upon IDL (Interactive Data 
Language), a commercial and closed-source interactive data analysis 
environment sold by Exelis, a global aerospace, defense, information 
and services company. The SSW gen package provides the base analysis 
capabilities such as time series analysis, time conversions, spectral 
fitting, image display, and file I/O. It is composed of 89\% IDL code 
while the remainder is mostly perl (5.4\%) and csh scripts (3.8\%). 
According to data generated using David A. Wheeler's 'SLOCCount', 
SSW/gen is composed of 289,724 lines of code with a total approximate 
value of \$10.4M (based on the Basic COCOMO model). Individual 
missions and instruments provide additional optional packages to 
support and process their data. For example, the SDO package 
\cite{sdo} is 100\% IDL code and is 11k lines of code (\$360k) 
and the SOHO package \cite{soho} is 100k lines of code (\$3M) and is 
99.9\% IDL code. SSW is hosted and distributed by the Lockheed Martin 
Solar and Astrophysics Laboratory.

%The RHESSI package is 99.97\% IDL code and is 145k total lines of code (\$5M)

While SSW is open-source and freely available, the IDL core. The cost 
of IDL is prohibitive for many undergraduates as well as institutions 
with limited means. 

One of SunPy's key aims is to provide a free and modern alternative 
to the SolarSoft (gen) library. This is made possible by the recent 
rise of the scientific Python community. Python is a general purpose 
high-level programming language. It is free to use, cross-platform 
and is both an accessible and powerful programming language. Python 
is one of the top ten most popular programming languages in the world 
according to the 2014 TIOBE Index 
\url{http://www.tiobe.com/index.php/content/paperinfo/tpci/index.html}.
It therefore has a very large user community which extends faroutside 
of the solar physics or astronomy communities. It is very extensible 
with C, C++, Fortran or even IDL. Many books and on-line 
documentation is available. For example at the time writing a search 
on Amazon for books on IDL programming generated $189$ results while 
a search for Python yielded $1,788$ results. Python has a strong 
emphasis on readability a concept which is well-aligned with the 
scientific endeavor.
Additionally, Python is increasingly being taught in universities,
which means that research projects can begin quickly 
without the need for students to learn a new (and relatively niche) 
language. Finally, knowledge of Python is a skill which is useful and well 
compensated outside of the scientific community (average salary of 
\$110k in the Washington DC area according to indeed.com at the time 
of writing). Python continues to see increased use in the astronomy 
community \cite{greenfield2011} which has similar goals and 
requirements as the solar physics community. Finally, Python plays 
well with many technologies such as web servers \cite{dolgert2008}, 
SQL databases, interactive and shareable notebook-based computing 
\cite{perez2007}. For all of these reasons, Python is an ideal 
platform for a modern solar physics data analysis environment.

SunPy is built on top of the core scientific Python packages namely 
\texttt{NumPy}, \texttt{SciPy}, \texttt{matplotlib}. These core 
packages provide capabilities surpassing that provided by IDL for 
array manipulation and 2-d plotting \cite{greenfield2011}. Providing 
similar capabilities as SSW/gen in Python is made easier by 
leveraging the diverse Python community. For example, SunPy makes use 
of the \texttt{Pandas} package to provide high-performance, 
easy-to-use data structures and data analysis tools for time series 
data\cite{mckinney2012}. \texttt{Pandas} was originally developed for 
quantitative analysis of financial data and has seen grown into a 
generalized data analysis package. SunPy also relies on and is 
coordinating with the \texttt{Astropy} package 
\cite{theastropycollaboration2013}. It is still in active development 
but already provides many relevant capabilities such as full-featured
support for FITS files, absolute dates and time conversions, units 
and unit conversions, and model-fitting. Pandas (134k lines of code) 
and AstroPy (90k lines of code) allow SunPy to leverage a total value 
of \$7.6M worth of development which is close to the \$10.4M for 
SSW/gen assuming a rough equivalency between a line of Python code 
and a line of IDL code. 

The design philosophy of SunPy is to provide a clean, simple to use, 
and well structured package that provides the \textit{core} tools for 
solar physics. The primary focus of SunPy's development early 
development is to provide specialised, linked, datatypes that allow 
the import, processing and visualisation of all types of solar data. 
These include \texttt{Map}, \texttt{LightCurve} and 
\texttt{Spectrum}, these cover spatial, timeseries and spectral data  
respectively. 

The purpose of this paper is to provide an overview of SunPy's 
current capabilities, an overview of the development model and 
community aspects of the SunPy project as well as future plans. The 
latest release of SunPy is available in PyPI and can be installed in 
the usual manner.