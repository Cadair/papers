\section{Introduction}\label{sec:Intro}

Science is driven by the analysis of data of growing variety and complexity.
Modern advances in sensor technology, combined with the availability of inexpensive 
storage, has led to rapid increases in the amount of data available to scientists in almost
every discipline.  Solar physics is no exception to this trend. For example,
NASA's \textit{Solar Dynamics Observatory} (\textit{SDO}) spacecraft, launched
in February 2010, produces over 1 TB of data per day \citep{aia}. Managing and
analysing this mountain of data requires increasingly sophisticated software
tools. These tools should be robust, easy to use and modify, have a transparent
development history, as well as conforming to modern software engineering
standards. Software with these qualities provide a strong foundation which can support the
needs of the community as data volumes grow and science questions evolve.

The SunPy project aims to provide a free, open-source and openly developed
software package for the analysis and visualisation of solar data. SunPy makes
use of Python and scientific Python packages. Python is a free, general-purpose, 
powerful and easy-to-learn high-level programming language. Additionally, Python is 
widely used outside of scientific fields in areas like `big data' analytics, web 
development and in educational environments. For example, \texttt{pandas} was 
originally developed for quantitative analysis of financial data and has since 
grown into a generalised time-series data-analysis package. Python continues to 
see increased use in the astronomy community \citep{greenfield2011}, which has 
similar goals and requirements as the solar physics community. Finally, Python 
integrates well with many technologies such as web servers \citep{dolgert2008} and databases. 

The development of a package such as SunPy is made possible by the rich ecosystem of 
scientific packages available in Python. Core packages such as \texttt{NumPy}, 
\texttt{SciPy} and \texttt{matplotlib} provide the basic functionality expected of a 
scientific programming language,
such as array manipulation, core numerical algorithms, and visualisation.
Building upon these foundations, packages such as \texttt{astropy}, \texttt{pandas} and
\texttt{scikit-image} provide more domain-specific functionality.

SunPy is designed to be a clean, simple-to-use, and well-structured package that provides 
the \textit{core} tools for solar data analysis, motivated by the need for a free and 
modern alternative to the existing SolarSoft (SSW) library. While SSW is open source 
and freely available, it relies on IDL (Interactive Data Language), a proprietary 
data-analysis environment.

The purpose of this paper is to provide an overview of SunPy's current
capabilities, an overview of the project's development model, community aspects of the
project, and future plans. The latest release of SunPy, version 0.4,
can be downloaded from \url{http://sunpy.org} or can be
installed using the Python package index (\url{http://pypi.python.org/pypi}).