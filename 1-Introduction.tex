\section{Introduction}\label{sec:Intro}

Science is driven by the analysis of data of an ever-growing variety and 
complexity. Advances in sensor technology, combined with the availability of 
inexpensive 
storage, have led to rapid increases in the amount of data available to scientists in almost
every discipline.  Solar physics is no exception to this trend. For example,
NASA's \textit{Solar Dynamics Observatory} (\textit{SDO}) spacecraft, launched
in February 2010, produces over 1 TB of data per day \citep{pesnell2012}. Managing and
analysing these data requires increasingly sophisticated software
tools. These tools should be robust, easy to use and modify, have a transparent
development history, and conform to modern software-engineering
standards. Software with these qualities provide a strong foundation that can support the
needs of the community as data volumes grow and science questions evolve.

%The SunPy project aims to provide a free, open-source, and openly developed
%software package for the analysis and visualisation of solar data.
The SunPy project aims to provide a software package with these qualities for 
the analysis and visualisation of solar data. SunPy makes
use of Python and scientific Python packages. Python is a free, general-purpose, 
powerful, and easy-to-learn high-level programming language. Additionally, Python is 
widely used outside of scientific fields in areas like `big data' analytics, web 
development, and educational environments. For example, \texttt{pandas} was 
originally developed for quantitative analysis of financial data and has since 
grown into a generalised time-series data-analysis package. Python continues to 
see increased use in the astronomy community \citep{greenfield2011}, which has 
similar goals and requirements as the solar physics community. Finally, Python 
integrates well with many technologies such as web servers \citep{dolgert2008} and databases. 

The development of a package such as SunPy is made possible by the rich ecosystem of 
scientific packages available in Python. Core packages such as \texttt{NumPy}, 
\texttt{SciPy} \citep{jones2001}, and \texttt{matplotlib} \citep{hunter2007} provide 
the basic functionality expected of a scientific programming language,
such as array manipulation, core numerical algorithms, and visualisation, respectively.
Building upon these foundations, packages such as \texttt{astropy} (astronomy)
\citep{theastropycollaboration2013}, \texttt{pandas} (time-series)
\citep{mckinney2010, mckinney2012}, and \texttt{scikit-image} (image processing)
\citep{vanderwalt2014} provide more domain-specific functionality.

A typical workflow begins with a solar physicist manually identifying
a small number of events of interest on the Sun.  This is typically
done in order to investigate in detail the physics of these events
(for example, the large solar flare of 23 July 2002 has Astrophysical
Journal Letters volume 595, dedicated to its analysis).
In this workflow, an event is investigated in depth which requires 
data from many different instruments.
These data are typically provided in many different formats (for
example, FITS (Flexible Image Transport System, \cite{refId0}), CSV, or
binary files), and contain many different types of data (such as
images, lightcurves and spectra).  In addition, the repositories these data reside
in can have different access methods.  This workflow is characterized
by the large number of heterogeneous datasets used in the
investigation of a small number of solar events.

Another typical workflow begins with the solar physicist identifying a
large sample of data or events.  The goal here is obtain information
about the population in general.  An example might be to calculate the
fractal dimension of a large number of active region magnetic fields
\citep{2005ApJ...631..628M}, or to calculate the observed temperatures
in a population of solar flares \citep{2012ApJS..202...11R}.  This
workflow is typically characterized by lower data heterogeneity, but
with a larger number of files.

The volume and variety of solar data used in these workflows drives
the need for an environment in which obtaining and performing common
solar-physics operations on these data is as simple and intuitive as
possible.  SunPy is designed to be a clean, simple-to-use, and
well-structured open-source package that provides the \textit{core}
tools for solar data analysis, motivated by the need for a free and
modern alternative to the existing SolarSoft (SSW) library
\citep{freeland1998}. While SSW is open source and freely available,
it relies on IDL (Interactive Data Language), a proprietary
data-analysis environment.

The purpose of this paper is to provide an overview of SunPy's current
capabilities, an overview of the project's development model, community aspects of the
project, and future plans. The latest release of SunPy, version 0.5,
can be downloaded from \url{http://sunpy.org} or can be
installed using the Python package index (\url{http://pypi.python.org/pypi}).
