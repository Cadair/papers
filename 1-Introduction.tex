\section{Introduction}
Modern solar physics and astrophysics, along with most of modern science is 
increasingly becoming data-driven. Where observational and numerical data is 
being produced at a faster rate than new hypotheses. For example, NASA's Solar 
Dynamics Observatory (SDO) satellite records over 1 TB of data per day all of 
which is telemetered to the ground and available for analysis.
Solar physics therefore requires increasingly complex software tools to enable 
scientific discovery by analysis of this data. To make meaningful and 
reproducible advances in our understanding of the Sun it is important that the 
software tools to be standardised, easy to use, and transparent, so that the 
community can build upon a common foundation.

SunPy aims to provide a free, open source and openly developed software package 
for the analysis and visualisation of solar data. SunPy makes use of the Python 
programming language and the breath of high-quality scientific focused packages 
available for it. The Python programming language is a general purpose, 
powerful and easy to learn high-level programming language. Python is one of 
the top ten most popular programming languages in the world according to the 
2014 TIOBE Index 
\footnote{\url{http://www.tiobe.com/index.php/content/paperinfo/tpci/index.html}},
 and is widely used outside of scientific fields in web development, education 
and even 3D game design.

The development of a package such as SunPy in Python is made possible by the 
rich scientific Python ecosystem of packages available. Core packages in this 
ecosystem such as \texttt{NumPy}, \texttt{SciPy} and \texttt{matplotlib} 
provide the basic functionality expected of a scientific programming language, 
\textit{i.e.} array manipulation, core numerical algorithms and visualisation. 
Ontop of this packages such as \texttt{Astropy}, \texttt{pandas} and 
\texttt{scikit-image} provide a tremendous amount of more domain-specific 
functionality.

The design philosophy of SunPy is to provide a clean, simple to use, and well 
structured package that provides the \textit{core} tools for solar physics. The 
primary focus of SunPy's early development is to provide specialised, linked, 
datatypes that allow the acquisition, processing and visualisation of all types 
of solar data.

The purpose of this paper is to provide an overview of SunPy's current 
capabilities, an overview of the development model and community aspects of the 
SunPy project as well as future plans. The latest release of SunPy is available 
in PyPI and can be installed in the usual manner for a Python package
\footnote{http://docs.sunpy.org/en/latest/sunpy/guide/installation/index.html}.