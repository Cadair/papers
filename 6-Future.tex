\section{Future of SunPy}\label{sec:future}
SunPy as a project has existed for roughly three years. In this time 
the code 
base has grown to over 110,000 lines. SunPy in its current form is a 
useful 
package for the analysis of calibrated solar data, however, the code 
base is 
still very changeable from release to release.
% RJH:  Again, lines of code is a weird measure.  Also, no way there is 110,000 lines that we
% actually wrote.  does this include docs, comments, and sphinx crap?

As discussed in Section \ref{sec:Intro}, the primary focus of the 
SunPy library is the analysis and visualisation of `high-level' solar 
data. This means data that has been put through instrument processing 
and 
calibration routines, and contains full (WCS) coordinate information. 
The plan for SunPy is to continue development within this 
scope. The 
primary components of this plan are to provide a set of data types 
that are 
interchangeable with one another: e.g., if you slice a 
\texttt{MapCube} 
along one spatial coordinate, a \texttt{LightCurve} of intensity along the 
time range of 
the \texttt{MapCube} should be returned. To achieve this goal, all the 
data 
types need to share a unified coordinate system architecture so that 
each data 
type is aware of what the physical type of its data is and how 
operations on 
that data should be performed.

In concert with the work on the data types, further integration with 
the 
\texttt{astropy} package will enable SunPy to inherit the large 
amount of work 
done in that project to support a platform for astronomical data 
analysis. For 
example, the \texttt{astropy.units} submodule provides a simple and 
efficient 
way of representing physical units in Python data types; there are 
plans to 
incorporate this widely in the SunPy code base. In the past 12 months, 
the 
collaboration with the Astropy project 
\citep{theastropycollaboration2013} and 
its community has grown, and this direct collaboration between Astropy 
and 
SunPy will be one of the key strengths of SunPy going forward.
