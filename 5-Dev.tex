\section{Development and Community}\label{sec:dev}
SunPy is a community-developed library, designed and developed for and by 
the solar physics community. Not only is all the source code publicly available 
online under the permissive 2-clause BSD licence, the whole 
development 
process is also online and open for anyone to contribute to.
SunPy's development makes use of the online 
service GitHub (\url{http://github.com}) and Git\footnote{\url{http://git-scm.com/}}
as its distributed version control software. 

The continued success of an open-source project depends on many factors;
three of the most important are (1) utility and quality of the code, (2) documentation, and (3) an
active community \cite{bangerth2013}. Several tools (some specific to Python) are used by
SunPy to make achieving these goals more accessible. To maintain high-quality code, a 
transparent and collaborative development workflow made possible by GitHub is used (described
further below).
Additionally, SunPy makes use of `continuous integration' provided by
Travis (\url{http://travis-ci.org}), a process by which the addition of any new code 
automatically triggers a comprehensive review of written unit tests. If any single test
fails then the community is alerted before the code is accepted. The unit test coverage is
also monitored by the \href{http://coveralls.io}{Coveralls} service.

Quality documentation is
one of the most important factors determining the success of any software project. 
Thankfully powerful tools already exist in Python to support this thanks to native
Python's focus on its own documentation. SunPy makes use of the Sphinx (\url{http://sphinx-doc.org})
documentation generator. Sphinx uses reStructuredText as its markup language: 
an easy-to-read, what-you-see-is-what-you-get plaintext markup syntax. It supports
many output formats most notably HTML as well as PDF, and ePub and provides a rich
hierarchically structured view of in-code documentation strings. The SunPy documentation 
is built automatically and hosted on Read the Docs (\url{http://readthedocs.org})
at \url{http://docs.sunpy.org}. 

In order to maintain an active community, communication is key.  The SunPy community makes
use of a number of channels to communicate. For immediate communications, an active IRC chat
room (\#SunPy) is hosted on freenode.net. For more involved or less immediate needs, such as
developer comments or discussion, a open mailing list is hosted on Google Groups. 
Discussion of bugs and bug tracking are provided directly on GitHub as well as new
code-review discussions.

Many of the larger features in SunPy have been developed 
with the support of both the ESA-SOCIS and GSOC. 
Initial development of SunPy's VSO and HEK 
implementations were funded by the SOCIS program as well as a prototyped GUI 
and N-Dimensional datatype implementations. SunPy was supported by GSOC in 2013 
through the Python Software Foundation (PSF). It supported development of the HELIO and 
database sub-packages.

\subsection{Development Workflow}
The development workflow that GitHub enables is simple and transparent. The 
primary source code repository \url{http://github.com/sunpy/sunpy} can be 
`forked' by anyone with a GitHub account who can then modify their fork as they 
wish. Upon completion of any modifications it is possible for a user to create 
a 'Pull Request' (\url{http://github.com/sunpy/sunpy/pulls}) to the main SunPy 
repository in order to incorporate their modification into the SunPy code base. 
Pull requests are reviewed by the SunPy community before being accepted. Community consensus
is required to add new features.

Pull requests are a very powerful tool to maintain code quality, as each line 
of code that has been modified is highlighted for review and can be commented on. 
In order to maintain high quality code the following conditions must typically be met 
before code is accepted.

\begin{enumerate}
	\item  The code must follow 	
	\href{http://www.python.org/dev/peps/pep-0008/}{PEP8} the Python style 
	guidelines, this keeps all the code in SunPy styled in a consistent manner and easy to read.
	
	\item All new features require documentation in the form of doc strings as well as user
	guides. 
	
	\item The code must contain significant unit test coverage.
\end{enumerate}

This kind of development model is widely used within the scientific Python 
community as well as by a wide variety of other projects both open and closed 
source.
