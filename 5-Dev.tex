\section{Development and Community}
SunPy is a community developed library, it is designed and developed for and by 
the solar physics community. Not only is all the source code publicly available 
online under the permissive 2-clause BSD licence 
\footnote{\url{http://opensource.org/licenses/BSD-2-Clause}}, the whole 
development 
process is also online and open for anyone to contribute to.
The primary media by which SunPy's development is facilitated is the online 
service GitHub \footnote{\url{http://github.com}} which makes use of the 
distributed 
version control software \href{http://git-scm.com/}{git}.

The development workflow that GitHub enables is simple and transparent. The 
primary source code repository \url{http://github.com/sunpy/sunpy} can be 
`forked' by anyone with a GitHub account who can then modify their fork as they 
wish. Upon completion of any modifications it is possible for a user to create 
a 'Pull Request' (\url{http://github.com/sunpy/sunpy/pulls}) to the main SunPy 
repository where it can be reviewed by anyone and then accepted by one of the 
development team who have commit rights to the main SunPy repository.

Pull requests are a very powerful tool to maintain code quality, as each line 
of code that has been modified is highlighted for review. For code to be 
accepted into the SunPy repository three primary things have to have been 
completed:

\begin{enumerate}
	\item  The code has to follow 	
	\href{http://www.python.org/dev/peps/pep-0008/}{PEP8} the Python style 
	guidelines, this keeps all the code in SunPy (and 99\% of all other Python 
	code) styled in a consistent manner and easy to read.
	
	\item The code must be documented. SunPy's documentation is generated by 
	the Sphinx package, which is automatically run and hosted by 
	\url{http://readthedocs.org}.
	
	\item The code must contain full unit test coverage. SunPy code is well 
	tested and coverage is increasing all the time. All SunPy's tests are 
	automatically run for each `Pull Request' and branch by the 
	\href{http://travis-ci.org/sunpy/sunpy}{Travis-CI }	service, and a test 
	coverage record is kept by 
	\href{http://coveralls.io/r/sunpy/sunpy}{Coveralls}
\end{enumerate}

This kind of development model is widely used within the scientific Python 
community as well as buy a wide variety of other projects both open and closed 
source.

As well as having a open source and open development model SunPy has a active 
community, with lots of both on-topic and off-topic chat on the \#SunPy IRC 
channel on freenode.net and on the mailing lists. The project always welcomes 
new contributions of any scale.

Many of the larger features added to SunPy during it's life have been added 
with the aid of both ESA's Summer Of Code In Space (SOCIS) program and Google's 
Summer of Code (GSOC) program. Initial development of SunPy's VSO and HEK 
implementations were funded by the SOCIS program as well as a prototyped GUI 
and N-Dimensional datatype implementations. 2013 was the first year that SunPy 
obtained GSOC funding, through the Python Software Foundation (PSF). The two 
projects that were undertaken was the development of a HELIO package and a 
package for integration between the HEK and VSO, allowing the download of data 
from VSO relating to a HEK query. As well as the Database module, which 
provides a single query-able interface between local data and the VSO.