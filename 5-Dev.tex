\section{Development and Community}\label{sec:dev}
SunPy is a community-developed library, designed and developed for and by 
the solar physics community. Not only is all the source code publicly available 
online under the permissive 2-clause BSD licence, the whole 
development process is also online and open for anyone to contribute to.
SunPy's development makes use of the online service 
GitHub (\url{http://github.com}) and Git\footnote{For more information see \url{http://git-scm.com/}}
as its distributed version control software. 

The continued success of an open-source project depends on many factors;
three of the most important are (1) utility and quality of the code, (2) documentation, and (3) an
active community \citep{bangerth2013}. Several tools, some specific to Python, are used by
SunPy to make achieving these goals more accessible. To maintain high-quality code, a 
transparent and collaborative development workflow made possible by GitHub is used.
The following conditions typically must be met before code is accepted.
\begin{enumerate}
	\item  The code must follow the
	PEP 8 Python style 
	guidelines (\url{http://www.python.org/dev/peps/pep-0008/}) to maintain consistency in the SunPy code.
	
	\item All new features require documentation in the form of doc strings as well as user
	guides. 
	
	\item The code must contain unit tests to verify that the code is behaving 
	as expected.

    \item Community consensus is reached that the new code is valuable and appropriately implemented.
\end{enumerate}
This kind of development model is widely used within the scientific Python 
community as well as by a wide variety of other projects, both open and closed 
source. 

Additionally, SunPy makes use of `continuous integration' provided by Travis CI (\url{http://travis-ci.org}), a process by which the addition of any new code 
automatically triggers a comprehensive review of the code functionality which are maintained as unit tests.
 If any single test
fails, the community is alerted before the code is accepted. The unit-test coverage is monitored by
a service called Coveralls (\url{http://coveralls.io}).

High-quality documentation is
one of the most important factors determining the success of any software project. 
Powerful tools already exist in Python to support documentation, thanks to native
Python's focus on its own documentation. SunPy makes use of the Sphinx (\url{http://sphinx-doc.org})
documentation generator. Sphinx uses reStructuredText as its markup language, which is
an easy-to-read, what-you-see-is-what-you-get plaintext markup syntax. It supports
many output formats most notably HTML, as well as PDF and ePub, and provides a rich,
hierarchically structured view of in-code documentation strings. The SunPy documentation 
is built automatically and is hosted by Read-the-Docs (\url{http://readthedocs.org})
at \url{http://docs.sunpy.org}. 

Communication is the key to maintaining an active community, and the SunPy community 
uses a number of different tools to facilitate communication. For immediate communications, an active IRC chat
room (\#SunPy) is hosted on freenode.net. For more involved or less immediate needs, such as
developer comments or discussions, an open mailing list is hosted by Google Groups. 
Bug tracking, code reviews, and feature-request discussions take place directly on GitHub.
The SunPy community also reaches out to the wider solar physics
community through presentations, functionality demonstrations, and informal meetups at scientific
meetings. 

In order to enable the long-term development of SunPy, a formal organizational
structure has been defined. The management of SunPy is the responsibility of 
the SunPy board, a group of elected members of the community. The board elects
a lead developer whose is responsible for the day to day development of SunPy.
SunPy also makes use of Python-style Enhancement proposals which can be proposed
by the community and are voted on by the board. These proposals set the overal
direction of SunPy's development.