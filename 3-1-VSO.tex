\subsection{VSO}\label{ssec:vso}

The Virtual Solar Observatory (VSO) was built to facilitate
access to solar data from the different archives around the 
world \cite{hill_virtual_2009}.  
Enabling therefore a single standard query interface to all
the available products from solar observation.  
VSO allows to query data not just by instrument or mission for 
a time range but by physical variables as for example:
intensity, magnetic field, particle flux or spectral range to name a few.
VSO's main interface is web-based\footnote{\url{http://virtualsolar.org}}, 
however a SOAP webservice is also available.
SunPy access to the VSO through the \texttt{vso} module which it uses 
the \texttt{suds} package to connect to the webservice.
This module was developed under the ESA's Summer of Code programme of 2010.

The \texttt{vso} module includes advanced features through the use
of VSO attributes objects.  
The time range, the instrument, the wavelength, the physical observable,
and many other properties are set as attributes objects which could
be combine using \texttt{and} - just by chaining together the arguments -
or \texttt{or} operators - by joining the attributes using the ``$|$'' operator.
Listing~\ref{code:vso_query} shows few examples of queries made using
such attributes.

\begin{listing}[h]
\begin{minted}{pycon}
>>> from sunpy.net import vso
>>> client=vso.VSOClient()
>>> tstart, tend = '2011/6/7 05:30', '2011/6/7 10:30'

Querying just a single instrument
>>> lasco_query = client.query(vso.attrs.Time(tstart, tend), 
...                            vso.attrs.Instrument('lasco'))
>>> lasco_query.num_records()
40
>>> lasco_query.show() 
Start time           End time             Source  Instrument  Type  
----------           --------             ------  ----------  ----  
2011-06-07 05:35:23  2011-06-07 05:35:48  SOHO    LASCO       CORONA
2011-06-07 05:43:09  2011-06-07 05:43:29  SOHO    LASCO       CORONA
...

Adding to the query STEREO's coronographs: COR1
>>> instrument_condition = ( vso.attrs.Instrument('lasco') | 
...                          vso.attrs.Detector('cor1') )
>>> coronographs = client.query(vso.attrs.Time(tstart, tend)), 
...                             instrument_condition)
>>> coronographs.num_records()
485
>>> coronographs.show() 
Start time           End time             Source    Instrument  Type  
----------           --------             ------    ----------  ----  
2011-06-07 05:35:23  2011-06-07 05:35:48  SOHO      LASCO       CORONA
2011-06-07 05:43:09  2011-06-07 05:43:29  SOHO      LASCO       CORONA
...
2011-06-07 10:25:43  2011-06-07 10:25:45  STEREO_B  SECCHI      CORONA
2011-06-07 10:30:00  2011-06-07 10:30:01  STEREO_A  SECCHI      CORONA
\end{minted}
\caption{Examples of different advanced queries through the \texttt{vso} 
  module using the attributes objects.}
\label{code:vso_query}
\end{listing}

The \texttt{vso} module also provides an interface to download the data,
allowing to customise the directory structure to save the data.
Listing~\ref{code:vso_down} exemplifies such case.


\begin{listing}[h]
\begin{minted}{pycon}
>>> pathformat = '/data/{instrument}/{detector}/{file}.fits'
>>> results = client.get(coronographs, 
...                      path = pathformat)

\end{minted}
\caption{From listing~\ref{code:vso_query}, this examples shows the 
  download method of the \texttt{vso} module.}
\label{code:vso_down}
\end{listing}
