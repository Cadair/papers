	\subsection{Map}
	Map is a 2D spatial data type, primarily used for images of the Sun and 
	inner heliosphere. It provides a wrapper around a data array (numpy 
	ndarray) and gives easy access to standard meta data in the header of the 
	image.
	The \texttt{Map} datatype provides convenience methods for many functions 
	such as, rotation and re-sampling as well as convenience visualisation 
	functions.
	The design of the map submodule is such that each instrument or 
	detector can subclass the parent \texttt{GenericMap} class to implement 
	special meta data handling or other data specific functions. Each subclass 
	of \texttt{GenericMap} can register with the \texttt{Map} factory class and 
	by implementing a method that returns \texttt{True} if the meta data 
	matches meta data for that instrument or detector, the \texttt{Map} factory 
	will automatically return an instance of the specific \texttt{GenericMap} 
	subclass. As of version 0.4 SunPy has \texttt{Map} specialisations for the 
	following instruments: \textit{IRIS} SJI frames, \textit{SDO/AIA} and 
	\textit{HMI}, \textit{SOHO/EIT} and 	\textit{LASCO}, 
	\textit{STEREO/EUVI} 	and \textit{COR}, \textit{YOHKOH/SXT}, 
	\textit{RHESSI}, \textit{PROBA2/SWAP} and\textit{HINODE/XRT}.
	As well as providing the base classes the map submodule provides two 
	collection classes, \texttt{CompositeMap} and \texttt{MapCube}, for 
	temporally and spatially aligned data respectively. \texttt{MapCube} 
	provides methods for animation of its series of \texttt{Map} objects. 
	\texttt{CompositeMap} provides methods for overlaying spatially aligned 
	data, with support for visualisation of images and contour lines overlaid 
	upon each other.