\subsection{HELIO}\label{ssec:helio}

The HELiophysics Integrated Observatory 
(HELIO, \url{http://helio-vo.eu}) has put together a list of web 
services which allows scientists to query and discover data throughout 
the heliosphere (\textit{e.g.}, solar, planetary, inter-planetary data, 
\ldots)\citep{dps2012}.
HELIO has been built with a Service-Oriented Architecture, 
\textit{i.e.} its capabilities are split into a number of tasks that are 
implemented as separate services. 
HELIO comprises nine different public services, which allows scientists
to search in different catalogues of registered events, solar features,
data from instruments in the heliosphere and other information like planets or 
spacecraft position in time. 
Additionally, HELIO provides a service that uses a propagation model to link 
the data in different points of the solar system by its original nature 
(\textit{e.g.}, Earth auroras are a signature of magnetic field disturbances 
produced few days before on the Sun).
In addition to the primary, web-based interface to HELIO
(\url{http://hfe.helio-vo.eu}), the services can be separately accessed.

SunPy's \texttt{hec} module provides an interface to the
\textit{HELIO Event Catalogue} (HEC) service, and this module was developed as
part of a Google Summer of Code project in 2013.
The HEC service currently provides access to 84 catalogues from different
sources (automated and manual classification, event lists from publications, \ldots).
As with all of the HELIO services, the HEC service provides results in VOTable 
data format (defined by IVOA \cite{ochsenbein_ivoa_2011}), and the \texttt{hec}
module parses this output using the \texttt{astropy.io.votable} package.
This format has the advantage of containing metadata with information like
data provenance and the performed query.

Listing~\ref{code:helio} shows an example of how to obtain information
from different catalogues of coronal mass ejections (CMEs).

\begin{listing}[h]
\begin{minted}[bgcolor=bg]{pycon}
>>> from sunpy.net.helio import hec
>>> hc = hec.Client()
>>> tstart, tend = '2011-06-07T06:00:00', '2011-06-07T12:00:00'
>>> event_type = 'cme'

# Obtain from all the catalogues these which name contain our event of interest
>>> catalogues = hc.get_table_names()
>>> catalogues_event = [l[0] for l in catalogues if event_type in l[0]]

# Query all the catalogues that comes from cactus
>>> results = [hc.time_query(tstart, tend, event) 
...            for event in catalogues_event if 'cactus' in event]
>>> print results
********** This need to be added!
\end{minted}
\caption{Example of querying the HEC service.}
\label{code:helio}
\end{listing}

Additional HELIO services will be supported in the future by SunPy.
