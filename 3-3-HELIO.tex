\subsection{HELIO}\label{ssec:helio}

The HELiophysics Integrated Observatory 
(HELIO)\footnote{\url{http://helio-vo.eu}} has put together a list of web 
services which allows the scientist to query them in order to discover data in 
the heliosphere (\textit{e.g.}, solar, planetary, inter-planetary data, 
...)\cite{dps2012}. 
HELIO has been built with a Service-Oriented Architecture, 
\textit{i.e.} its capabilities are split into a number of tasks that are 
implemented as separate services. 
HELIO counts with up to nine different public services, which allows to the 
scientist to search in different catalogues of registered events, solar features,
data from instruments in the heliosphere and other information like planets or 
spacecraft position in time. 
Additionally, HELIO provides of a service that uses a propagation model to link 
the data in different points of the solar system by its original nature 
(\textit{e.g.}, Earth auroras are a signature of magnetic field disturbances 
produced few days before on the Sun).
The main interface to access to HELIO is through its front-end\footnote{
\url{http://hfe.helio-vo.eu}},
but since the system is based in webservices they can be accessed from many different tools.

The current version of SunPy has includes an interface to access the 
\textit{HELIO Event Catalogue} (HEC) service developed as part of Google
Summer of Code project in 2013.
This service provides at the moment access to 84 catalogues from different
sources (automated and manual classification, event lists from publications, \ldots).
HEC results (as the rest of HELIO services) provides the results in VOTable 
data format (defined by IVOA \cite{ochsenbein_ivoa_2011})
which the \texttt{astropy.io.votable} can interact with.
This format has the advantage of contain metadata with information like
data provenance and the performed query.

On listing~\ref{code:helio} we show some an example on how to obtain information
from different \textit{Coronal Mass Ejections}(CMEs) catalogues.

\begin{listing}[h]
\begin{minted}{pycon}
>>> from sunpy.net.helio import hec
>>> hc = hec.Client()

Desired time range and event type
>>> tstart = '2011-06-07T06:00:00'
>>> tend = '2011-06-07T12:00:00'
>>> event_type = 'cme'

Obtain from all the catalogues these which name contain our event of interest
>>> catalogues = hc.get_table_names()
>>> catalogues_event = [l[0] for l in catalogues if event_type in l[0]]
>>> print catalogues_event
******** This need to be added!

Query all the catalogues that comes from cactus
>>> results = [hc.time_query(tstart, tend, event) 
...            for event in catalogues_event if 'cactus' in event]
>>> print results
********** This need to be added!
\end{minted}
\caption{Demonstration of the \texttt{helio.hec} interface.}
\label{code:helio}
\end{listing}

It is in the SunPy roadmap to add support from the other HELIO services and improve
the usability of them towards a seamlessly interaction of this tool with the rest of
the environment, as for example with data visualisation or the database module.
