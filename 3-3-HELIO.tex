\subsection{HELIO}\label{ssec:helio}

The HELiophysics Integrated Observatory (HELIO)\footnote{For more information 
see \url{http://helio-vo.eu}} has 
compiled a list of web services which allows scientists to query and 
discover data throughout the heliosphere, from solar and magnetospheric data to planetary and 
inter-planetary data \citep{perez-suarez2012}.
HELIO is built with a Service-Oriented Architecture, 
i.e., its capabilities are divided into a number of tasks that are 
implemented as separate services. 
HELIO is made up of nine different public services, 
which allows scientists to search different catalogues of registered events, 
solar features, data from instruments in the heliosphere, and other information 
such as planetary or spacecraft position in time. 
Additionally, HELIO provides a service that uses a 
propagation model to link the data in different points of the solar system by 
its original nature (e.g., Earth auroras are a signature of magnetic 
field disturbances produced a few days before on the Sun).
In addition to the primary, web-based interface to 
HELIO, its services are available via an API.

SunPy's \texttt{hec} module provides an interface to the
HELIO Event Catalogue (HEC) service. 
This module was developed as
part of a Google Summer of Code (GSOC) project in 2013.
The HEC service currently provides access to 84 catalogues from different
sources.
As with all of the HELIO services, the HEC service provides results in VOTable 
data format (defined by IVOA, see \citealt{ochsenbein2011}).
The \texttt{hec} module parses this output using the \texttt{astropy.io.votable} package.
This format has the advantage of containing metadata with information like
data provenance and the performed query.

For example, Listing~\ref{code:helio} shows how to obtain information
from different catalogues of coronal mass ejections (CMEs).

\begin{listing}[h]
\pythoncode{pycode_helio.txt}
\caption{Example of querying the HEC service to multiple CME
catalogues, in this case the ones detected automatically 
by the Computer Aided CME Tracking feature recognition algorithm CACTus (\url{http://sidc.oma.be/cactus/}) \citep{robbrecht_automated_2009}.}
\label{code:helio}
\end{listing}