\subsection{HELIO}\label{ssec:helio}

The HELiophysics Integrated Observatory (\href{http://helio-vo.eu}{HELIO}) has 
compiled a list of web services which allows scientists to query and 
discover data throughout the heliosphere, from solar and magnetospheric data to planetary and 
inter-planetary data \citep{dps2012}.
HELIO is built with a Service-Oriented Architecture, 
i.e., its capabilities are divided into a number of tasks that are 
implemented as separate services. 
\href{http://helio-vo.eu}{HELIO} comprises nine different public services, 
which allows scientists to search different catalogues of registered events, 
solar features, data from instruments in the heliosphere and other information 
such as planetary or spacecraft position in time. 
Additionally, \href{http://helio-vo.eu}{HELIO} provides a service that uses a 
propagation model to link the data in different points of the solar system by 
its original nature (e.g., Earth auroras are a signature of magnetic 
field disturbances produced few days before on the Sun).
In addition to the primary, web-based interface to 
\href{http://helio-vo.eu}{HELIO}, the services can be separately accessed.

SunPy's \texttt{hec} module provides an interface to the
\textit{HELIO Event Catalogue} (HEC) service. This module was developed as
part of a Google Summer of Code (GSOC) project in 2013.
The HEC service currently provides access to 84 catalogues from different
sources.
As with all of the HELIO services, the HEC service provides results in VOTable 
data format (defined by IVOA \cite{ochsenbein_ivoa_2011}). The \texttt{hec}
module parses this output using the \texttt{astropy.io.votable} package.
This format has the advantage of containing metadata with information like
data provenance and the performed query.

For example, Listing~\ref{code:helio} shows how to obtain information
from different catalogues of coronal mass ejections (CMEs).

\begin{listing}[h]
\begin{minted}[bgcolor=bg]{pycon}
>>> from sunpy.net.helio import hec
>>> hc = hec.Client()
>>> tstart, tend = '2011-06-07T06:00:00', '2011-06-07T12:00:00'
>>> event_type = 'cme'

# From all the catalogues these which name contain our event of interest
>>> catalogues = hc.get_table_names()
>>> catalogues_event = [l[0] for l in catalogues 
...                     if event_type in l[0] and 'list' not in l[0]]

# Query all the catalogues that comes from cactus
>>> results = [hc.time_query(tstart, tend, event) 
...            for event in catalogues_event if 'cactus' in event]
>>> for cat in results:
...     print "{cat} has {nres} results".format(cat = cat.ID, \
...                                             nres = len(cat.array))
__helio_hec-cactus_stereoa_cme has 4 results
__helio_hec-cactus_stereob_cme has 3 results
__helio_hec-cactus_soho_cme has 7 results
\end{minted}
\caption{Example of querying the HEC service to multiple cme
catalogues, in this case the ones detected by \href{http://sidc.oma.be/cactus/}{CACTus}.}
\label{code:helio}
\end{listing}
% RJH: CACTus should be defined and a reference provided in the reference section.

In the future, additional HELIO services will be supported in the future by SunPy.
% RJH: This sentence is an orphan and should have a home constructed for it.
