\subsection{HEK}\label{ssec:hek}

The Sun is an active star and exhibits a wide range of transient phenomena 
(e.g., flares, radio bursts, coronal mass ejections) at many different time-scales, 
length-scales, and 
wavelengths. Observations and metadata concerning these phenomena are collected 
in the Heliophysics Event Knowledgebase (HEK) \citep{hurlburt2012}.  Entries are generated both by
automated algorithms and human observers.  Some of the information in the HEK 
reproduces feature and event data from elsewhere (for example, the \textit{GOES} flare catalogue),
and some is generated by the Solar Dynamics Observatory Feature Finding Team 
\citep{martens2012}.  A key feature of the HEK is that it
provides an homogeneous and well-described interface to a large amount of 
feature and event information. SunPy 
accesses this information through the \texttt{hek} module.  The \texttt{hek} module makes use of the 
HEK public API\footnote{For more information see \url{http://vso.stanford.edu/hekwiki/ApplicationProgrammingInterface}}.

Simple HEK queries consist of start time, an end time, and an event type 
(see Listing~\ref{code:hek:simple}). Event types are specified as upper case, 
two letter strings, and these strings are 
identical to the two letter abbreviations defined by HEK 
(see \url{http://www.lmsal.com/hek/VOEvent_Spec.html}). Users can see a
complete list and description of these abbreviations by looking at the documentation
for \texttt{hek.attrs.EventType}.

\begin{listing}[H]
\pythoncode{pycode_hek.txt}
\caption{Example usage of the \texttt{hek} module showing a simple HEK search for solar flares
on 2011 August 9.}
\label{code:hek:simple}
\end{listing}

Short-cuts are also provided for some often-used event types. For example, 
the flare attribute can be declared as either 
\texttt{hek.attrs.EventType("FL")} or as \texttt{hek.attrs.FL}. 

HEK attributes differ from VSO attributes (Section \ref{ssec:vso}) in that many 
of them are wrappers that conveniently expose comparisons by overloading Python 
operators. This allows filtering of the HEK entries by the properties of the 
event. As was mentioned above, the HEK stores feature and event metadata obtained 
in different ways, known generally as feature recognition methods (FRMs). 
The example in Listing~\ref{code:hek:frm} repeats the previous 
HEK query (see Listing \ref{code:hek:simple}), with an additional filter enabled 
to return only those events that have the FRM `SSW Latest Events'.  
Multiple comparisons can be made by including more comma-separated
conditions on the attributes in the call to the HEK query method.

\begin{listing}[H]
\pythoncode{pycode_hek2.txt}
\caption{An HEK query that returns only those flares that were
  detected by the `SSW Latest Events' feature recognition method.}
\label{code:hek:frm}
\end{listing}

HEK comparisons can be combined using Python's logical operators (e.g., \texttt{and}
and \texttt{or}). The ability to use comparison and logical operators on HEK attributes allows 
the construction of queries of arbitrary complexity.
For the query in Listing \ref{code:hek:or} returns
returns flares with helioprojective x coordinates west of 50 arcseconds or 
those that have a peak flux above 1000.0 in units defined by the FRM.

\begin{listing}[H]
\pythoncode{pycode_hek3.txt}
\caption{HEK query using the logical \texttt{or} operator.}
\label{code:hek:or}
\end{listing}
All FRMs report the required feature attributes, but the optional attributes 
are FRM dependent.  If a FRM does not have one of the optional attributes, 
\texttt{None} is returned by the \texttt{hek} module. 
 
Frequently after users have found events of interest the next step is to 
download observational data. The \texttt{H2VClient} module makes this
easier by providing a translation layer between HEK query results
and VSO data queries. This capability is demonstrated in Listing~\ref{code:hek2vso}.
\begin{listing}[H]
\pythoncode{pycode_hek4.txt}
\caption{Code snippet continuing from Listing~\ref{code:hek:or} showing the 
query and download of data from the first HEK result from the VSO.}
\label{code:hek2vso}
\end{listing}
