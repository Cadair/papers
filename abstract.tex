\begin{abstract}
This paper presents version 0.4 of SunPy, a community-developed Python package 
for solar physics and data analysis.

Python, a free, cross platform, general purpose, high-level programming language, 
has seen widespread adoption among the scientific community resulting in the availability 
of a large range of software, from numerical computation (NumPy, SciPy) and machine 
learning to spectral analysis and visualisation (Matplotlib). SunPy is a data analysis 
toolkit specialising in providing the software necessary to analyse solar and heliospheric 
datasets in Python. It aims to provide a free and open-source alternative to the 
IDL-based SolarSoft (SSW) solar data analysis environment. We present the latest 
release of SunPy (0.4). SunPy provides downloading capability through integration with the Virtual Solar 
Observatory (VSO) and the the Heliophysics Event Knowledgebase (HEK). It can open 
image fits files from major solar missions (SDO/AIA, SOHO/EIT, SOHO/LASCO, STEREO) 
into WCS-aware maps. SunPy provides advanced time-series tools for data from mission 
such as GOES, SDO/EVE, and Proba2/LYRA as well as support for radio spectra (e.g. e-Callisto). 
We present examples of solar data analysis in SunPy, and show how Python-based solar 
data-analysis can leverage the many existing data analysis tools already available in 
Python. We discuss the future goals of the project and encourage interested users to 
become involved in the planning and development of SunPy.

% schriste - we need an abstract!
% just pasted in old abstract as place holder but still needs work
\end{abstract}
