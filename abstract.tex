\begin{abstract}
This paper presents version 0.4 of SunPy, a community-developed Python package 
for solar physics and data analysis.
Python, a free, cross platform, general purpose, high-level programming 
language, has seen widespread adoption among the scientific community resulting 
in the availability of a large range of available software, from numerical 
computation (NumPy, SciPy) and machine learning (scikit-learn) to spectral 
analysis and visualisation (Matplotlib).
SunPy is a data analysis toolkit specialising in providing the software 
necessary to analyse solar and heliospheric datasets in Python. 
SunPy is open source software (BSD licence) and has a open and transparent 
development workflow that anyone can contribute to.
SunPy provides access to solar data through integration with the Virtual 
Solar Observatory (VSO), the the Heliophysics Event Knowledgebase (HEK) and the 
HELIO webservices. It can open data files from major solar missions (e.g. 
SDO, SOHO, STEREO and IRIS) into WCS-aware `Maps'. Time-series data from 
mission such as GOES, SDO/EVE, and Proba2/LYRA into `LightCurves' and radio 
spectra from e-Callisto or STEREO/SWAVES into `Spectrograms'. We present 
examples of solar data analysis in SunPy, and show how Python-based solar 
data-analysis can leverage the many existing data analysis tools already 
available in Python. We discuss the future goals of the project and encourage 
interested users to become involved in the planning and development of SunPy.
\end{abstract}