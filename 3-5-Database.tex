\subsection{The SunPy Database}
% TODO: reference / footnote to Summer of Code website?
The database package has been added as a Google Summer of Code (GSoC) project
in 2013. Its purpose is to offer a database interface to store and manage data
that has been retrieved via packages from the net package. Note that currently
only the VSO package is supported, not the HEK package.

The need for a database package arose because using VSO alone led to two major
problems:
\begin{enumerate}
  \item downloaded data cannot be queried or browsed
  \item to fetch data that has already been saved on the local hard drive,
    the VSO has to be queried again
\end{enumerate}

Here is a list of the main features:
\begin{itemize}
  \item detects if certain query has already been used before and queries
    the database then instead of the VSO
  % TODO: reference to SA docs? or list all currently supported SQL dialects here?
  \item many supported SQL dialects
  % TODO: footnotes for LRU/LFU or some explanation in parens behind them?
  \item database may act as a Cache (algorithm may be LRU, LFU or a custom one)
  \item all writing operations can be undone and redone
\end{itemize}

Using the database package, one uses the Database class to connect with a
database and perform operations on it.

\begin{listing}
\begin{minted}{python}
>>> from sunpy.net import vso
>>> from sunpy.database import Database
>>> database = Database('sqlite:///')
>>> database.download(
...     vso.attrs.Time((2011, 9, 20, 1), (2011, 9, 20, 2)),
...     vso.attrs.Instrument('RHESSI'))
>>> len(database)
6
\end{minted}
\caption{Connecting to a database and adding new entries.}
\label{code:db_1}
\end{listing}

In the code listing \ref{code:db_1}, a connection to new SQLite in-memory
database is established. After that, the attributes from the \textsc{VSO}
package are used to query the \textsc{VSO} and download the corresponding
files. The function \texttt{len} can be used to get the number of all database
records.

Note that the number of new database entries does not depend on the
number of files being downloaded but on the total number of \textsc{FITS}
headers in the files! If the used query had been used to query the \textsc{VSO}
directly, the number of query results would have been 2. You can see in the
following code example, listing \ref{code:db_2}, that indeed the entries can
be grouped to two results.

The function \texttt{display\_entries} displays an iterable of database entries
in a nicely-formatted \textsc{ASCII} table. The headlines correspond to the
attributes of the respective database entries.

\begin{listing}
\begin{minted}{python}
>>> from sunpy.database.tables import display_entries
>>> print display_entries(
...     database,
...     ['id', 'observation_time_start', 'wavemin', 'wavemax', 'download_time'])
id observation_time_start wavemin        wavemax           download_time             
-- ---------------------- -------        -------           -------------             
1  2011-09-20 01:09:20    0.413280643067 7.29318781883e-05 2014-01-18 19:21:35.525397
2  2011-09-20 01:09:20    0.413280643067 7.29318781883e-05 2014-01-18 19:21:35.525469
3  2011-09-20 01:09:20    0.413280643067 7.29318781883e-05 2014-01-18 19:21:35.525532
4  2011-09-19 23:33:40    0.413280643067 7.29318781883e-05 2014-01-18 19:24:53.429665
5  2011-09-19 23:33:40    0.413280643067 7.29318781883e-05 2014-01-18 19:24:53.429731
6  2011-09-19 23:33:40    0.413280643067 7.29318781883e-05 2014-01-18 19:24:53.429792
\end{minted}
\caption{Displaying database entries in a table.}
\label{code:db_2}
\end{listing}

As said before, the database can be used as a cache. By default, the size of
the cache is not limited which in other words means that caching is disabled.
To limit the number of database entries to a certain number a posteriori, use
the method \texttt{set\_cache\_size} and pass it the number of maximum entries.
The default algorithm to remove entries from the cache is \textsc{LRU}
(least recently used), i.e. the item whose usage is farthest away is removed
first.  % FIXME is 'whose' correct here? ^

\begin{listing}
\begin{minted}{python}
>>> database.set_cache_size(3)
>>> print display_entries(
...     database,
...     ['id', 'observation_time_start', 'wavemin', 'wavemax', 'download_time'])
id observation_time_start wavemin        wavemax           download_time             
-- ---------------------- -------        -------           -------------             
4  2011-09-19 23:33:40    0.413280643067 7.29318781883e-05 2014-01-18 19:24:53.429665
5  2011-09-19 23:33:40    0.413280643067 7.29318781883e-05 2014-01-18 19:24:53.429731
6  2011-09-19 23:33:40    0.413280643067 7.29318781883e-05 2014-01-18 19:24:53.429792
\end{minted}
\caption{Displaying database entries in a table.}
\label{code:db_3}
\end{listing}

In code listing \ref{code:db_3} the entries with the IDs \#1 -- \#3 have been
removed because they have been added first and are therefore the entries that
have been used least recently.
